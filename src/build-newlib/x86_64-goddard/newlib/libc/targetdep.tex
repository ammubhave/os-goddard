@node Stdlib
@chapter Standard Utility Functions (@file{stdlib.h})

This chapter groups utility functions useful in a variety of programs.
The corresponding declarations are in the header file @file{stdlib.h}.

@menu 
* _Exit::       End program execution without cleaning up
* a64l::        String to long long
* abort::       Abnormal termination of a program
* abs::         Integer absolute value (magnitude)
* assert::      Macro for Debugging Diagnostics
* atexit::      Request execution of functions at program exit
* atof::        String to double or float
* atoi::        String to integer
* atoll::       String to long long
* bsearch::	Binary search
* calloc::      Allocate space for arrays
* div::         Divide two integers
* ecvtbuf::     Double or float to string of digits
* ecvt::        Double or float to string of digits (malloc result)
* __env_lock::		Lock environment list for getenv and setenv
* gvcvt::       Format double or float as string
* exit::        End program execution
* getenv::      Look up environment variable
* labs::        Long integer absolute value (magnitude)
* ldiv::        Divide two long integers
* llabs::       Long long integer absolute value (magnitude)
* lldiv::       Divide two long long integers
* malloc::      Allocate and manage memory (malloc, realloc, free)
* mallinfo::	Get information about allocated memory
* __malloc_lock::	Lock memory pool for malloc and free
* mbsrtowcs::	Convert a character string to a wide-character string
* mbstowcs::	Minimal multibyte string to wide string converter
* mblen::	Minimal multibyte length
* mbtowc::      Minimal multibyte to wide character converter
* on_exit::     Request execution of functions at program exit
* qsort::	Array sort
* rand::        Pseudo-random numbers
* rand48::      Uniformly distributed pseudo-random numbers
* strtod::      String to double or float
* strtol::      String to long
* strtoll::     String to long long
* strtoul::     String to unsigned long
* strtoull::    String to unsigned long long
* wcsrtombs::	Convert a wide-character string to a character string
* wcstod::      Wide string to double or float
* wcstol::      Wide string to long
* wcstoll::     Wide string to long long
* wcstoul::     Wide string to unsigned long
* wcstoull::    Wide string to unsigned long long
* system::      Execute command string
* wcstombs::	Minimal wide string to multibyte string converter
* wctomb::      Minimal wide character to multibyte converter
@end menu

@page
@include stdlib/_Exit.def

@page
@include stdlib/a64l.def

@page
@include stdlib/abort.def

@page
@include stdlib/abs.def

@page
@include stdlib/assert.def

@page
@include stdlib/atexit.def

@page
@include stdlib/atof.def

@page
@include stdlib/atoi.def

@page
@include stdlib/atoll.def

@page
@include search/bsearch.def

@page
@include stdlib/calloc.def

@page
@include stdlib/div.def

@page
@include stdlib/efgcvt.def

@page
@include stdlib/ecvtbuf.def

@page
@include stdlib/envlock.def

@page
@include stdlib/exit.def

@page
@include stdlib/getenv.def

@page
@include stdlib/labs.def

@page
@include stdlib/ldiv.def

@page
@include stdlib/llabs.def

@page
@include stdlib/lldiv.def

@page
@include stdlib/malloc.def

@page
@include stdlib/mstats.def

@page
@include stdlib/mlock.def

@page
@include stdlib/mblen.def

@page
@include stdlib/mbsnrtowcs.def

@page
@include stdlib/mbstowcs.def

@page
@include stdlib/mbtowc.def

@page
@include stdlib/on_exit.def

@page
@include search/qsort.def

@page
@include stdlib/rand.def

@page
@include stdlib/rand48.def

@page
@include stdlib/strtod.def

@page
@include stdlib/strtol.def

@page
@include stdlib/strtoll.def

@page
@include stdlib/strtoul.def

@page
@include stdlib/strtoull.def

@page
@include stdlib/wcsnrtombs.def

@page
@include stdlib/wcstod.def

@page
@include stdlib/wcstol.def

@page
@include stdlib/wcstoll.def

@page
@include stdlib/wcstoul.def

@page
@include stdlib/wcstoull.def

@page
@include stdlib/system.def

@page
@include stdlib/wcstombs.def

@page
@include stdlib/wctomb.def

@node Ctype
@chapter Character Type Macros and Functions (@file{ctype.h})
This chapter groups macros (which are also available as subroutines)
to classify characters into several categories (alphabetic,
numeric, control characters, whitespace, and so on), or to perform
simple character mappings.

The header file @file{ctype.h} defines the macros.
@menu 
* isalnum::   Alphanumeric character predicate
* isalpha::   Alphabetic character predicate
* isascii::   ASCII character predicate
* iscntrl::   Control character predicate
* isdigit::   Decimal digit predicate
* islower::   Lowercase character predicate
* isprint::   Printable character predicates (isprint, isgraph)
* ispunct::   Punctuation character predicate
* isspace::   Whitespace character predicate
* isupper::   Uppercase character predicate
* isxdigit::  Hexadecimal digit predicate
* toascii::   Force integers to ASCII range
* tolower::   Translate characters to lowercase
* toupper::   Translate characters to uppercase
* iswalnum::  Alphanumeric wide character predicate
* iswalpha::  Alphabetic wide character predicate
* iswblank::  Blank wide character predicate
* iswcntrl::  Control wide character predicate
* iswdigit::  Decimal digit wide character predicate
* iswgraph::  Graphic wide character predicate
* iswlower::  Lowercase wide character predicate
* iswprint::  Printable wide character predicate
* iswpunct::  Punctuation wide character predicate
* iswspace::  Whitespace wide character predicate
* iswupper::  Uppercase wide character predicate
* iswxdigit:: Hexadecimal digit wide character predicate
* iswctype::  Extensible wide-character test
* wctype::    Compute wide-character test type
* towlower::  Translate wide characters to lowercase
* towupper::  Translate wide characters to uppercase
* towctrans:: Extensible wide-character translation
* wctrans::   Compute wide-character translation type
@end menu

@page
@include ctype/isalnum.def

@page
@include ctype/isalpha.def

@page
@include ctype/isascii.def

@page
@include ctype/iscntrl.def

@page
@include ctype/isdigit.def

@page
@include ctype/islower.def

@page
@include ctype/isprint.def

@page
@include ctype/ispunct.def

@page
@include ctype/isspace.def

@page
@include ctype/isupper.def

@page
@include ctype/isxdigit.def

@page
@include ctype/toascii.def

@page
@include ctype/tolower.def

@page
@include ctype/toupper.def

@page
@include ctype/iswalnum.def

@page
@include ctype/iswalpha.def

@page
@include ctype/iswcntrl.def

@page
@include ctype/iswblank.def

@page
@include ctype/iswdigit.def

@page
@include ctype/iswgraph.def

@page
@include ctype/iswlower.def

@page
@include ctype/iswprint.def

@page
@include ctype/iswpunct.def

@page
@include ctype/iswspace.def

@page
@include ctype/iswupper.def

@page
@include ctype/iswxdigit.def

@page
@include ctype/iswctype.def

@page
@include ctype/wctype.def

@page
@include ctype/towlower.def

@page
@include ctype/towupper.def

@page
@include ctype/towctrans.def

@page
@include ctype/wctrans.def


@node Stdio
@chapter Input and Output (@file{stdio.h})

This chapter comprises functions to manage files
or other input/output streams. Among these functions are subroutines
to generate or scan strings according to specifications from a format string.

The underlying facilities for input and output depend on the host
system, but these functions provide a uniform interface.

The corresponding declarations are in @file{stdio.h}.

The reentrant versions of these functions use macros

@example
_stdin_r(@var{reent})
_stdout_r(@var{reent})
_stderr_r(@var{reent})
@end example

@noindent
instead of the globals @code{stdin}, @code{stdout}, and
@code{stderr}.  The argument <[reent]> is a pointer to a reentrancy
structure.

@menu
* clearerr::    Clear file or stream error indicator
* diprintf::    Print to a file descriptor (integer only)
* dprintf::     Print to a file descriptor
* fclose::      Close a file
* fcloseall::   Close all files
* fdopen::      Turn an open file into a stream
* feof::        Test for end of file
* ferror::      Test whether read/write error has occurred
* fflush::      Flush buffered file output
* fgetc::       Get a character from a file or stream
* fgetpos::     Record position in a stream or file
* fgets::       Get character string from a file or stream
* fgetwc::      Get a wide character from a file or stream
* fgetws::      Get a wide character string from a file or stream
* fileno::      Get file descriptor associated with stream
* fmemopen::    Open a stream around a fixed-length buffer
* fopen::       Open a file
* fopencookie:: Open a stream with custom callbacks
* fpurge::      Discard all pending I/O on a stream
* fputc::       Write a character on a stream or file
* fputs::       Write a character string in a file or stream
* fputwc::      Write a wide character to a file or stream
* fputws::      Write a wide character string to a file or stream
* fread::       Read array elements from a file
* freopen::     Open a file using an existing file descriptor
* fseek::       Set file position
* fsetpos::     Restore position of a stream or file
* ftell::       Return position in a stream or file
* funopen::     Open a stream with custom callbacks
* fwide::	Set and determine the orientation of a FILE stream
* fwrite::      Write array elements from memory to a file or stream
* getc::        Get a character from a file or stream (macro)
* getc_unlocked::	Get a character from a file or stream (macro)
* getchar::     Get a character from standard input (macro)
* getchar_unlocked::	Get a character from standard input (macro)
* getdelim::    Get character string from a file or stream
* getline::     Get character string from a file or stream
* gets::        Get character string from standard input (obsolete)
* getw::        Get a word (int) from a file or stream
* getwchar::    Get a wide character from standard input
* mktemp::      Generate unused file name
* open_memstream::	Open a write stream around an arbitrary-length buffer
* perror::      Print an error message on standard error
* putc::        Write a character on a stream or file (macro)
* putc_unlocked::	Write a character on a stream or file (macro)
* putchar::     Write a character on standard output (macro)
* putchar_unlocked::	Write a character on standard output (macro)
* puts::        Write a character string on standard output
* putw::        Write a word (int) to a file or stream
* putwchar::    Write a wide character to standard output
* remove::      Delete a file's name
* rename::      Rename a file
* rewind::      Reinitialize a file or stream
* setbuf::      Specify full buffering for a file or stream
* setbuffer::   Specify full buffering for a file or stream with size
* setlinebuf::  Specify line buffering for a file or stream
* setvbuf::     Specify buffering for a file or stream
* siprintf::    Write formatted output (integer only)
* siscanf::     Scan and format input (integer only)
* sprintf::     Write formatted output
* sscanf::      Scan and format input
* swprintf::    Write formatted wide character output
* swscanf::     Scan and format wide character input
* tmpfile::     Create a temporary file
* tmpnam::      Generate name for a temporary file
* ungetc::      Push data back into a stream
* ungetwc::     Push wide character data back into a stream
* vfprintf::    Format variable argument list
* vfscanf::     Scan variable argument list
* vfwprintf::   Format variable wide character argument list
* vfwscanf::    Scan and format argument list from wide character input
* viprintf::    Format variable argument list (integer only)
* viscanf::     Scan variable format list (integer only)
@end menu

@page
@include stdio/clearerr.def

@page
@include stdio/diprintf.def

@page
@include stdio/dprintf.def

@page
@include stdio/fclose.def

@page
@include stdio/fcloseall.def

@page
@include stdio/fdopen.def

@page
@include stdio/feof.def

@page
@include stdio/ferror.def

@page
@include stdio/fflush.def

@page
@include stdio/fgetc.def

@page
@include stdio/fgetpos.def

@page
@include stdio/fgets.def

@page
@include stdio/fgetwc.def

@page
@include stdio/fgetws.def

@page
@include stdio/fileno.def

@page
@include stdio/fmemopen.def

@page
@include stdio/fopen.def

@page
@include stdio/fopencookie.def

@page
@include stdio/fpurge.def

@page
@include stdio/fputc.def

@page
@include stdio/fputs.def

@page
@include stdio/fputwc.def

@page
@include stdio/fputws.def

@page
@include stdio/fread.def

@page
@include stdio/freopen.def

@page
@include stdio/fseek.def

@page
@include stdio/fsetpos.def

@page
@include stdio/ftell.def

@page
@include stdio/funopen.def

@page
@include stdio/fwide.def

@page
@include stdio/fwrite.def

@page
@include stdio/getc.def

@page
@include stdio/getc_u.def

@page
@include stdio/getchar.def

@page
@include stdio/getchar_u.def

@page
@include stdio/getdelim.def

@page
@include stdio/getline.def

@page
@include stdio/gets.def

@page
@include stdio/getw.def

@page
@include stdio/getwchar.def

@page
@include stdio/mktemp.def

@page
@include stdio/open_memstream.def

@page
@include stdio/perror.def

@page
@include stdio/putc.def

@page
@include stdio/putc_u.def

@page
@include stdio/putchar.def

@page
@include stdio/putchar_u.def

@page
@include stdio/puts.def

@page
@include stdio/putw.def

@page
@include stdio/putwchar.def

@page
@include stdio/remove.def

@page
@include stdio/rename.def

@page
@include stdio/rewind.def

@page
@include stdio/setbuf.def

@page
@include stdio/setbuffer.def

@page
@include stdio/setlinebuf.def

@page
@include stdio/setvbuf.def

@page
@include stdio/siprintf.def

@page
@include stdio/siscanf.def

@page
@include stdio/sprintf.def

@page
@include stdio/sscanf.def

@page
@include stdio/swprintf.def

@page
@include stdio/swscanf.def

@page
@include stdio/tmpfile.def

@page
@include stdio/tmpnam.def

@page
@include stdio/ungetc.def

@page
@include stdio/ungetwc.def

@page
@include stdio/vfprintf.def

@page
@include stdio/vfscanf.def

@page
@include stdio/vfwprintf.def

@page
@include stdio/vfwscanf.def

@page
@include stdio/viprintf.def

@page
@include stdio/viscanf.def
@node Strings
@chapter Strings and Memory (@file{string.h})

This chapter describes string-handling functions and functions for
managing areas of memory.  The corresponding declarations are in
@file{string.h}.

@menu
* bcmp::        Compare two memory areas
* bcopy::       Copy memory regions
* bzero::       Initialize memory to zero
* index::       Search for character in string
* memccpy::     Copy memory regions up to end-token
* memchr::      Find character in memory
* memcmp::      Compare two memory areas
* memcpy::      Copy memory regions
* memmem::      Find memory segment
* memmove::     Move possibly overlapping memory
* mempcpy::	Copy memory regions and locate end
* memset::      Set an area of memory
* rindex::      Reverse search for character in string
* stpcpy::      Copy string returning a pointer to its end
* stpncpy::     Counted copy string returning a pointer to its end
* strcasecmp::	Compare strings ignoring case
* strcasestr::	Find string segment ignoring case
* strcat::      Concatenate strings
* strchr::      Search for character in string
* strcmp::      Character string compare
* strcoll::     Locale-specific character string compare
* strcpy::      Copy string
* strcspn::     Count chars not in string
* strerror::    Convert error number to string
* strerror_r::  Convert error number to string
* strlen::      Character string length
* strlwr::	Convert string to lowercase
* strncasecmp::	Compare strings ignoring case
* strncat::     Concatenate strings
* strncmp::     Character string compare
* strncpy::     Counted copy string
* strnlen::     Character string length
* strpbrk::     Find chars in string
* strrchr::     Reverse search for character in string
* strsignal::	Return signal message string
* strspn::      Find initial match
* strstr::      Find string segment
* strtok::      Get next token from a string
* strupr::	Convert string to upper case
* strxfrm::     Transform string
* swab::        Swap adjacent bytes
* wcscasecmp::  Compare wide character strings ignoring case
* wcsdup::      Wide character string duplicate
* wcsncasecmp:: Compare wide character strings ignoring case
@end menu

@page
@include string/bcmp.def

@page
@include string/bcopy.def

@page
@include string/bzero.def

@page
@include string/index.def

@page
@include string/memccpy.def

@page
@include string/memchr.def

@page
@include string/memcmp.def

@page
@include string/memcpy.def

@page
@include string/memmem.def

@page
@include string/memmove.def

@page
@include string/mempcpy.def

@page
@include string/memset.def

@page
@include string/rindex.def

@page
@include string/stpcpy.def

@page
@include string/stpncpy.def

@page
@include string/strcasecmp.def

@page
@include string/strcasestr.def

@page
@include string/strcat.def

@page
@include string/strchr.def

@page
@include string/strcmp.def

@page
@include string/strcoll.def

@page
@include string/strcpy.def

@page
@include string/strcspn.def

@page
@include string/strerror.def

@page
@include string/strerror_r.def

@page
@include string/strlen.def

@page
@include string/strlwr.def

@page
@include string/strncasecmp.def

@page
@include string/strncat.def

@page
@include string/strncmp.def

@page
@include string/strncpy.def

@page
@include string/strnlen.def

@page
@include string/strpbrk.def

@page
@include string/strrchr.def

@page
@include string/strsignal.def

@page
@include string/strspn.def

@page
@include string/strstr.def

@page
@include string/strtok.def

@page
@include string/strupr.def

@page
@include string/strxfrm.def

@page
@include string/swab.def

@page
@include string/wcscasecmp.def

@page
@include string/wcsdup.def

@page
@include string/wcsncasecmp.def
@node Wchar strings
@chapter Wide Character Strings (@file{wchar.h})

This chapter describes wide-character string-handling functions and
managing areas of memory containing wide characters.  The corresponding 
declarations are in @file{wchar.h}.

@menu
* wmemchr::     Find wide character in memory
* wmemcmp::     Compare two wide-character memory areas
* wmemcpy::     Copy wide-character memory regions
* wmemmove::    Move possibly overlapping wide-character memory
* wmemset::     Set an area of memory to a specified wide character
* wcscat::      Concatenate wide-character strings
* wcschr::      Search for wide character in string
* wcscmp::      Wide-character string compare
* wcscoll::     Locale-specific wide-character string compare
* wcscpy::      Copy wide-character string
* wcpcpy::      Copy a wide-character string returning a pointer to its end
* wcscspn::     Count wide characters not in string
* wcsftime::    Convert date and time to a formatted wide-character string
* wcslcat::     Concatenate wide-character strings to specified length
* wcslcpy::     Copy wide-character string to specified length
* wcslen::      Wide-character string length
* wcsncat::     Concatenate wide-character strings
* wcsncmp::     Wide-character string compare
* wcsncpy::     Counted copy wide-character string
* wcpncpy::     Copy part of a wide-character string returning a pointer to its end
* wcsnlen::     Wide-character string length with maximum limit
* wcspbrk::     Find wide characters in string
* wcsrchr::     Reverse search for wide character in string
* wcsspn::      Find initial match in wide-character string
* wcsstr::      Find wide-character string segment
* wcstok::      Tokenize wide-character string
* wcswidth::    Number of column positions of a wide-character string
* wcsxfrm::     Locale-specific wide-character string transformation
* wcwidth::     Number of column positions of a wide-character code
@end menu

@page
@include string/wmemchr.def

@page
@include string/wmemcmp.def

@page
@include string/wmemcpy.def

@page
@include string/wmemmove.def

@page
@include string/wmemset.def

@page
@include string/wcscat.def

@page
@include string/wcschr.def

@page
@include string/wcscmp.def

@page
@include string/wcscoll.def

@page
@include string/wcscpy.def

@page
@include string/wcpcpy.def

@page
@include string/wcscspn.def

@page
@include time/wcsftime.def

@page
@include string/wcslcat.def

@page
@include string/wcslcpy.def

@page
@include string/wcslen.def

@page
@include string/wcsncat.def

@page
@include string/wcsncmp.def

@page
@include string/wcsncpy.def

@page
@include string/wcpncpy.def

@page
@include string/wcsnlen.def

@page
@include string/wcspbrk.def

@page
@include string/wcsrchr.def

@page
@include string/wcsspn.def

@page
@include string/wcsstr.def

@page
@include string/wcstok.def

@page
@include string/wcswidth.def

@page
@include string/wcsxfrm.def

@page
@include string/wcwidth.def

@node Signals
@chapter Signal Handling (@file{signal.h})

A @dfn{signal} is an event that interrupts the normal flow of control
in your program.  Your operating environment normally defines the full
set of signals available (see @file{sys/signal.h}), as well as the
default means of dealing with them---typically, either printing an
error message and aborting your program, or ignoring the signal.

All systems support at least the following signals:
@table @code
@item SIGABRT
Abnormal termination of a program; raised by the <<abort>> function.

@item SIGFPE
A domain error in arithmetic, such as overflow, or division by zero.

@item SIGILL
Attempt to execute as a function data that is not executable.

@item SIGINT
Interrupt; an interactive attention signal.

@item SIGSEGV
An attempt to access a memory location that is not available.

@item SIGTERM
A request that your program end execution.
@end table

Two functions are available for dealing with asynchronous
signals---one to allow your program to send signals to itself (this is
called @dfn{raising} a signal), and one to specify subroutines (called
@dfn{handlers} to handle particular signals that you anticipate may
occur---whether raised by your own program or the operating environment.

To support these functions, @file{signal.h} defines three macros:

@table @code
@item SIG_DFL
Used with the @code{signal} function in place of a pointer to a
handler subroutine, to select the operating environment's default
handling of a signal.

@item SIG_IGN
Used with the @code{signal} function in place of a pointer to a
handler, to ignore a particular signal.

@item SIG_ERR
Returned by the @code{signal} function in place of a pointer to a
handler, to indicate that your request to set up a handler could not
be honored for some reason.
@end table

@file{signal.h} also defines an integral type, @code{sig_atomic_t}.
This type is not used in any function declarations; it exists only to
allow your signal handlers to declare a static storage location where
they may store a signal value.  (Static storage is not otherwise
reliable from signal handlers.)

@menu
* psignal:: Print a signal message to standard error
* raise::   Send a signal
* signal::  Specify handler subroutine for a signal
@end menu

@page
@include signal/psignal.def

@page
@include signal/raise.def

@page
@include signal/signal.def
@node Timefns
@chapter Time Functions (@file{time.h})

This chapter groups functions used either for reporting on time
(elapsed, current, or compute time) or to perform calculations based
on time.

The header file @file{time.h} defines three types.  @code{clock_t} and
@code{time_t} are both used for representations of time particularly
suitable for arithmetic.  (In this implementation, quantities of type
@code{clock_t} have the highest resolution possible on your machine,
and quantities of type @code{time_t} resolve to seconds.)  @code{size_t}
is also defined if necessary for quantities representing sizes. 

@file{time.h} also defines the structure @code{tm} for the traditional
representation of Gregorian calendar time as a series of numbers, with
the following fields: 

@table @code
@item tm_sec
Seconds, between 0 and 60 inclusive (60 allows for leap seconds).

@item tm_min
Minutes, between 0 and 59 inclusive.

@item tm_hour
Hours, between 0 and 23 inclusive.

@item tm_mday
Day of the month, between 1 and 31 inclusive.

@item tm_mon
Month, between 0 (January) and 11 (December).

@item tm_year
Year (since 1900), can be negative for earlier years.

@item tm_wday
Day of week, between 0 (Sunday) and 6 (Saturday).

@item tm_yday
Number of days elapsed since last January 1, between 0 and 365 inclusive.

@item tm_isdst
Daylight Savings Time flag: positive means DST in effect, zero means DST
not in effect, negative means no information about DST is available.
Although for mktime(), negative means that it should decide if DST is in
effect or not.
@end table

@menu
* asctime::     Format time as string
* clock::       Cumulative processor time
* ctime::       Convert time to local and format as string
* difftime::    Subtract two times
* gmtime::      Convert time to UTC (GMT) traditional representation
* localtime::   Convert time to local representation
* mktime::      Convert time to arithmetic representation
* strftime::    Convert date and time to a user-formatted string
* time::        Get current calendar time (as single number)
* __tz_lock::   Lock time zone global variables
* tzset::       Set timezone info
@end menu

@page
@include time/asctime.def

@page
@include time/clock.def

@page
@include time/ctime.def

@page
@include time/difftime.def

@page
@include time/gmtime.def

@page
@include time/lcltime.def

@page
@include time/mktime.def

@page
@include time/strftime.def

@page
@include time/time.def

@page
@include time/tzlock.def

@page
@include time/tzset.def
@node Locale
@chapter Locale (@file{locale.h})

A @dfn{locale} is the name for a collection of parameters (affecting
collating sequences and formatting conventions) that may be different
depending on location or culture.  The @code{"C"} locale is the only
one defined in the ANSI C standard.

This is a minimal implementation, supporting only the required @code{"C"}
value for locale; strings representing other locales are not
honored.  (@code{""} is also accepted; it represents the default locale
for an implementation, here equivalent to @code{"C"}.


@file{locale.h} defines the structure @code{lconv} to collect the
information on a locale, with the following fields:

@table @code
@item char *decimal_point
The decimal point character used to format ``ordinary'' numbers (all
numbers except those referring to amounts of money).  @code{"."} in the
C locale. 

@item char *thousands_sep
The character (if any) used to separate groups of digits, when
formatting ordinary numbers.
@code{""} in the C locale.

@item char *grouping
Specifications for how many digits to group (if any grouping is done at
all) when formatting ordinary numbers.  The @emph{numeric value} of each
character in the string represents the number of digits for the next
group, and a value of @code{0} (that is, the string's trailing
@code{NULL}) means to continue grouping digits using the last value
specified.  Use @code{CHAR_MAX} to indicate that no further grouping is
desired.  @code{""} in the C locale. 

@item char *int_curr_symbol
The international currency symbol (first three characters), if any, and
the character used to separate it from numbers.
@code{""} in the C locale.

@item char *currency_symbol
The local currency symbol, if any.
@code{""} in the C locale.

@item char *mon_decimal_point
The symbol used to delimit fractions in amounts of money.
@code{""} in the C locale.

@item char *mon_thousands_sep
Similar to @code{thousands_sep}, but used for amounts of money.
@code{""} in the C locale.

@item char *mon_grouping
Similar to @code{grouping}, but used for amounts of money.
@code{""} in the C locale.

@item char *positive_sign
A string to flag positive amounts of money when formatting.
@code{""} in the C locale.

@item char *negative_sign
A string to flag negative amounts of money when formatting.
@code{""} in the C locale.

@item char int_frac_digits
The number of digits to display when formatting amounts of money to
international conventions.
@code{CHAR_MAX} (the largest number representable as a @code{char}) in
the C locale. 

@item char frac_digits
The number of digits to display when formatting amounts of money to
local conventions.
@code{CHAR_MAX} in the C locale. 

@item char p_cs_precedes
@code{1} indicates the local currency symbol is used before a
@emph{positive or zero} formatted amount of money; @code{0} indicates
the currency symbol is placed after the formatted number.
@code{CHAR_MAX} in the C locale. 

@item char p_sep_by_space
@code{1} indicates the local currency symbol must be separated from
@emph{positive or zero} numbers by a space; @code{0} indicates that it
is immediately adjacent to numbers.
@code{CHAR_MAX} in the C locale. 

@item char n_cs_precedes
@code{1} indicates the local currency symbol is used before a
@emph{negative} formatted amount of money; @code{0} indicates
the currency symbol is placed after the formatted number.
@code{CHAR_MAX} in the C locale. 

@item char n_sep_by_space
@code{1} indicates the local currency symbol must be separated from
@emph{negative} numbers by a space; @code{0} indicates that it
is immediately adjacent to numbers.
@code{CHAR_MAX} in the C locale. 

@item char p_sign_posn
Controls the position of the @emph{positive} sign for
numbers representing money.  @code{0} means parentheses surround the
number; @code{1} means the sign is placed before both the number and the
currency symbol; @code{2} means the sign is placed after both the number
and the currency symbol; @code{3} means the sign is placed just before
the currency symbol; and @code{4} means the sign is placed just after
the currency symbol.
@code{CHAR_MAX} in the C locale. 

@item char n_sign_posn
Controls the position of the @emph{negative} sign for numbers
representing money, using the same rules as @code{p_sign_posn}.
@code{CHAR_MAX} in the C locale. 
@end table

@menu
* setlocale::  Select or query locale
@end menu

@page
@include locale/locale.def
@node Reentrancy
@chapter Reentrancy

@cindex reentrancy
Reentrancy is a characteristic of library functions which allows multiple
processes to use the same address space with assurance that the values stored
in those spaces will remain constant between calls. The Red Hat
newlib implementation of the library functions ensures that 
whenever possible, these library functions are reentrant.  However,
there are some functions that can not be trivially made reentrant.
Hooks have been provided to allow you to use these functions in a fully
reentrant fashion.

@findex _reent
@findex reent.h
@cindex reentrancy structure
These hooks use the structure @code{_reent} defined in @file{reent.h}.
A variable defined as @samp{struct _reent} is called a @dfn{reentrancy
structure}.  All functions which must manipulate global information are
available in two versions.  The first version has the usual name, and
uses a single global instance of the reentrancy structure.  The second
has a different name, normally formed by prepending @samp{_} and
appending @samp{_r}, and takes a pointer to the particular reentrancy
structure to use.

For example, the function @code{fopen} takes two arguments, @var{file}
and @var{mode}, and uses the global reentrancy structure.  The function
@code{_fopen_r} takes the arguments, @var{struct_reent}, which is a
pointer to an instance of the reentrancy structure, @var{file}
and @var{mode}.	

There are two versions of @samp{struct _reent}, a normal one and one
for small memory systems, controlled by the @code{_REENT_SMALL}
definition from the (automatically included) @file{<sys/config.h>}.

@cindex global reentrancy structure
@findex _impure_ptr
Each function which uses the global reentrancy structure uses the global
variable @code{_impure_ptr}, which points to a reentrancy structure.

This means that you have two ways to achieve reentrancy.  Both require
that each thread of execution control initialize a unique global
variable of type @samp{struct _reent}:

@enumerate
@item
@cindex extra argument, reentrant fns
Use the reentrant versions of the library functions, after initializing
a global reentrancy structure for each process.  Use the pointer to this
structure as the extra argument for all library functions.

@item
Ensure that each thread of execution control has a pointer to its own
unique reentrancy structure in the global variable @code{_impure_ptr},
and call the standard library subroutines.
@end enumerate

@cindex list of reentrant functions
@cindex reentrant function list
The following functions are provided in both reentrant
and non-reentrant versions.

@example
@exdent @emph{Equivalent for errno variable:}
_errno_r

@exdent @emph{Locale functions:}
_localeconv_r  _setlocale_r

@exdent @emph{Equivalents for stdio variables:}
_stdin_r        _stdout_r       _stderr_r

@page  
@exdent @emph{Stdio functions:}
_fdopen_r       _perror_r       _tempnam_r
_fopen_r        _putchar_r      _tmpnam_r
_getchar_r      _puts_r         _tmpfile_r
_gets_r         _remove_r       _vfprintf_r
_iprintf_r      _rename_r       _vsnprintf_r
_mkstemp_r      _snprintf_r     _vsprintf_r
_mktemp_t       _sprintf_r

@exdent @emph{Signal functions:}
_init_signal_r  _signal_r
_kill_r         __sigtramp_r
_raise_r

@exdent @emph{Stdlib functions:}
_calloc_r       _mblen_r        _setenv_r
_dtoa_r         _mbstowcs_r     _srand_r
_free_r         _mbtowc_r       _strtod_r
_getenv_r       _memalign_r     _strtol_r
_mallinfo_r     _mstats_r       _strtoul_r
_malloc_r       _putenv_r       _system_r
_malloc_r       _rand_r         _wcstombs_r
_malloc_stats_r _realloc_r      _wctomb_r

@exdent @emph{String functions:}
_strdup_r       _strtok_r

@exdent @emph{System functions:}
_close_r        _link_r         _unlink_r
_execve_r       _lseek_r        _wait_r
_fcntl_r        _open_r         _write_r 
_fork_r         _read_r
_fstat_r        _sbrk_r
_gettimeofday_r _stat_r
_getpid_r       _times_r

@ifset STDIO64
@exdent @emph{Additional 64-bit I/O System functions:}
_fstat64_r	_lseek64_r	_open64_r
@end ifset

@exdent @emph{Time function:}
_asctime_r
@end example
@node Misc
@chapter Miscellaneous Macros and Functions
This chapter describes miscellaneous routines not covered elsewhere.

@menu 
* ffs::      Return first bit set in a word
* unctrl::   Return printable representation of a character
@end menu

@page
@include misc/ffs.def

@page
@include misc/unctrl.def
@c                                           -*- Texinfo -*-
@node Syscalls
@chapter System Calls

@cindex linking the C library
The C subroutine library depends on a handful of subroutine calls for
operating system services.  If you use the C library on a system that
complies with the POSIX.1 standard (also known as IEEE 1003.1), most of
these subroutines are supplied with your operating system.

If some of these subroutines are not provided with your system---in
the extreme case, if you are developing software for a ``bare board''
system, without an OS---you will at least need to provide do-nothing
stubs (or subroutines with minimal functionality) to allow your
programs to link with the subroutines in @code{libc.a}.

@menu
* Stubs::		Definitions for OS interface
* Reentrant Syscalls::	Reentrant covers for OS subroutines
@end menu

@node Stubs
@section Definitions for OS interface
@cindex stubs

@cindex subroutines for OS interface
@cindex OS interface subroutines
This is the complete set of system definitions (primarily subroutines)
required; the examples shown implement the minimal functionality
required to allow @code{libc} to link, and fail gracefully where OS
services are not available.  

Graceful failure is permitted by returning an error code.  A minor
complication arises here: the C library must be compatible with
development environments that supply fully functional versions of these
subroutines.  Such environments usually return error codes in a global
@code{errno}.  However, the Red Hat newlib C library provides a @emph{macro}
definition for @code{errno} in the header file @file{errno.h}, as part
of its support for reentrant routines (@pxref{Reentrancy,,Reentrancy}).

@cindex @code{errno} global vs macro
The bridge between these two interpretations of @code{errno} is
straightforward: the C library routines with OS interface calls
capture the @code{errno} values returned globally, and record them in
the appropriate field of the reentrancy structure (so that you can query
them using the @code{errno} macro from @file{errno.h}).

This mechanism becomes visible when you write stub routines for OS
interfaces.   You must include @file{errno.h}, then disable the macro,
like this:

@example
#include <errno.h>
#undef errno
extern int errno;
@end example

@noindent
The examples in this chapter include this treatment of @code{errno}.

@ftable @code
@item _exit
Exit a program without cleaning up files.  If your system doesn't
provide this, it is best to avoid linking with subroutines that require
it (@code{exit}, @code{system}).

@item close
Close a file.  Minimal implementation:

@example
int close(int file) @{
  return -1;
@}
@end example

@item environ
A pointer to a list of environment variables and their values.  For a
minimal environment, this empty list is adequate:

@example
char *__env[1] = @{ 0 @};
char **environ = __env;
@end example

@item execve
Transfer control to a new process.  Minimal implementation (for a system
without processes):

@example
#include <errno.h>
#undef errno
extern int errno;
int execve(char *name, char **argv, char **env) @{
  errno = ENOMEM;
  return -1;
@}
@end example

@item fork
Create a new process.  Minimal implementation (for a system without processes):

@example
#include <errno.h>
#undef errno
extern int errno;
int fork(void) @{
  errno = EAGAIN;
  return -1;
@}
@end example

@item fstat
Status of an open file.  For consistency with other minimal
implementations in these examples, all files are regarded as character
special devices.  The @file{sys/stat.h} header file required is
distributed in the @file{include} subdirectory for this C library.

@example
#include <sys/stat.h>
int fstat(int file, struct stat *st) @{
  st->st_mode = S_IFCHR;
  return 0;
@}
@end example

@item getpid
Process-ID; this is sometimes used to generate strings unlikely to
conflict with other processes.  Minimal implementation, for a system
without processes:

@example
int getpid(void) @{
  return 1;
@}
@end example

@item isatty
Query whether output stream is a terminal.   For consistency with the
other minimal implementations, which only support output to
@code{stdout}, this minimal implementation is suggested:

@example
int isatty(int file) @{
  return 1;
@}
@end example

@item kill
Send a signal.  Minimal implementation:

@example
#include <errno.h>
#undef errno
extern int errno;
int kill(int pid, int sig) @{
  errno = EINVAL;
  return -1;
@}
@end example

@item link
Establish a new name for an existing file.  Minimal implementation:

@example
#include <errno.h>
#undef errno
extern int errno;
int link(char *old, char *new) @{
  errno = EMLINK;
  return -1;
@}
@end example

@item lseek
Set position in a file.  Minimal implementation:

@example
int lseek(int file, int ptr, int dir) @{
  return 0;
@}
@end example

@item open
Open a file.  Minimal implementation:

@example
int open(const char *name, int flags, int mode) @{
  return -1;
@}
@end example

@item read
Read from a file.  Minimal implementation:

@example
int read(int file, char *ptr, int len) @{
  return 0;
@}
@end example

@item sbrk
Increase program data space.  As @code{malloc} and related functions
depend on this, it is useful to have a working implementation.  The
following suffices for a standalone system; it exploits the symbol
@code{_end} automatically defined by the GNU linker.

@example
@group
caddr_t sbrk(int incr) @{
  extern char _end;		/* @r{Defined by the linker} */
  static char *heap_end;
  char *prev_heap_end;
 
  if (heap_end == 0) @{
    heap_end = &_end;
  @}
  prev_heap_end = heap_end;
  if (heap_end + incr > stack_ptr) @{
    write (1, "Heap and stack collision\n", 25);
    abort ();
  @}

  heap_end += incr;
  return (caddr_t) prev_heap_end;
@}
@end group
@end example

@item stat
Status of a file (by name).  Minimal implementation:

@example
int stat(char *file, struct stat *st) @{
  st->st_mode = S_IFCHR;
  return 0;
@}
@end example

@item times
Timing information for current process.  Minimal implementation:

@example
int times(struct tms *buf) @{
  return -1;
@}
@end example

@item unlink
Remove a file's directory entry.  Minimal implementation:

@example
#include <errno.h>
#undef errno
extern int errno;
int unlink(char *name) @{
  errno = ENOENT;
  return -1; 
@}
@end example

@item wait
Wait for a child process.  Minimal implementation:
@example
#include <errno.h>
#undef errno
extern int errno;
int wait(int *status) @{
  errno = ECHILD;
  return -1;
@}
@end example

@item write
Write to a file.  @file{libc} subroutines will use this
system routine for output to all files, @emph{including}
@code{stdout}---so if you need to generate any output, for example to a
serial port for debugging, you should make your minimal @code{write}
capable of doing this.  The following minimal implementation is an
incomplete example; it relies on a @code{outbyte} subroutine (not
shown; typically, you must write this in assembler from examples
provided by your hardware manufacturer) to actually perform the output.

@example
@group
int write(int file, char *ptr, int len) @{
  int todo;

  for (todo = 0; todo < len; todo++) @{
    outbyte (*ptr++);
  @}
  return len;
@}
@end group
@end example

@end ftable

@page
@node Reentrant Syscalls
@section Reentrant covers for OS subroutines

Since the system subroutines are used by other library routines that
require reentrancy, @file{libc.a} provides cover routines (for example,
the reentrant version of @code{fork} is @code{_fork_r}).  These cover
routines are consistent with the other reentrant subroutines in this
library, and achieve reentrancy by using a reserved global data block
(@pxref{Reentrancy,,Reentrancy}).

@c FIXME!!! The following ignored text specifies how this section ought
@c to work;  however, both standalone info and Emacs info mode fail when
@c confronted with nodes beginning `_' as of 24may93.  Restore when Info
@c readers fixed!
@ignore
@menu
* _open_r::	Reentrant version of open
* _close_r::	Reentrant version of close
* _lseek_r::	Reentrant version of lseek
* _read_r::	Reentrant version of read
* _write_r::	Reentrant version of write
* _link_r::     Reentrant version of link
* _unlink_r::   Reentrant version of unlink
* _stat_r::     Reentrant version of stat
* _fstat_r::    Reentrant version of fstat
* _sbrk_r::     Reentrant version of sbrk
* _fork_r::	Reentrant version of fork
* _wait_r::	Reentrant version of wait
@end menu

@down
@include reent/filer.def
@include reent/execr.def
@include reent/statr.def
@include reent/fstatr.def
@include reent/linkr.def
@include reent/unlinkr.def
@include reent/sbrkr.def
@up
@end ignore

@ftable @code
@item _open_r
A reentrant version of @code{open}.  It takes a pointer
to the global data block, which holds @code{errno}.

@example
int _open_r(void *@var{reent},
    const char *@var{file}, int @var{flags}, int @var{mode});
@end example

@ifset STDIO64
@item _open64_r
A reentrant version of @code{open64}.  It takes a pointer
to the global data block, which holds @code{errno}.

@example
int _open64_r(void *@var{reent},
    const char *@var{file}, int @var{flags}, int @var{mode});
@end example
@end ifset

@item _close_r
A reentrant version of @code{close}.  It takes a pointer to the global
data block, which holds @code{errno}.

@example
int _close_r(void *@var{reent}, int @var{fd});
@end example

@item _lseek_r
A reentrant version of @code{lseek}.  It takes a pointer to the global
data block, which holds @code{errno}.

@example
off_t _lseek_r(void *@var{reent},
    int @var{fd}, off_t @var{pos}, int @var{whence});
@end example

@ifset STDIO64
@item _lseek64_r
A reentrant version of @code{lseek64}.  It takes a pointer to the global
data block, which holds @code{errno}.

@example
off_t _lseek64_r(void *@var{reent},
    int @var{fd}, off_t @var{pos}, int @var{whence});
@end example
@end ifset

@item _read_r
A reentrant version of @code{read}.  It takes a pointer to the global
data block, which holds @code{errno}.

@example
long _read_r(void *@var{reent},
    int @var{fd}, void *@var{buf}, size_t @var{cnt});
@end example

@item _write_r
A reentrant version of @code{write}.  It takes a pointer to the global
data block, which holds @code{errno}.

@example
long _write_r(void *@var{reent},
    int @var{fd}, const void *@var{buf}, size_t @var{cnt});
@end example

@item _fork_r
A reentrant version of @code{fork}.  It takes a pointer to the global
data block, which holds @code{errno}.

@example
int _fork_r(void *@var{reent});
@end example

@item _wait_r
A reentrant version of @code{wait}.  It takes a pointer to the global
data block, which holds @code{errno}.

@example
int _wait_r(void *@var{reent}, int *@var{status});
@end example

@item _stat_r
A reentrant version of @code{stat}.  It takes a pointer to the global
data block, which holds @code{errno}.

@example
int _stat_r(void *@var{reent},
    const char *@var{file}, struct stat *@var{pstat});
@end example

@item _fstat_r
A reentrant version of @code{fstat}.  It takes a pointer to the global
data block, which holds @code{errno}.

@example
int _fstat_r(void *@var{reent},
    int @var{fd}, struct stat *@var{pstat});
@end example

@ifset STDIO64
@item _fstat64_r
A reentrant version of @code{fstat64}.  It takes a pointer to the global
data block, which holds @code{errno}.

@example
int _fstat64_r(void *@var{reent},
    int @var{fd}, struct stat *@var{pstat});
@end example
@end ifset

@item _link_r
A reentrant version of @code{link}.  It takes a pointer to the global
data block, which holds @code{errno}.

@example
int _link_r(void *@var{reent},
    const char *@var{old}, const char *@var{new});
@end example

@item _unlink_r
A reentrant version of @code{unlink}.  It takes a pointer to the global
data block, which holds @code{errno}.

@example
int _unlink_r(void *@var{reent}, const char *@var{file});
@end example

@item _sbrk_r
A reentrant version of @code{sbrk}.  It takes a pointer to the global
data block, which holds @code{errno}.

@example
char *_sbrk_r(void *@var{reent}, size_t @var{incr});
@end example
@end ftable
